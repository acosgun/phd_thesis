\documentclass[12pt]{gatech-thesis}
\usepackage{amsmath,amssymb,latexsym,float,epsfig,subfigure}

%%
%% This example is adapted from ucthesis.tex, a part of the
%% UCTHESIS class package...
%%
\title{People Aware Mobile Robot Navigation} %% If you want to specify a linebreak
                               %% in the thesis title, you MUST use
                               %% \protect\\ instead of \\, as \\ is a
                               %% fragile command that \MakeUpperCase
                               %% will break!
\author{Akansel Cosgun}
\department{College of Computing}

%% Can have up to six readers, plus principaladvisor and
%% committeechair. All have the form
%%
%%  \reader{Name}[Department][Institution]
%%
%% The second and third arguments are optional, but if you wish to
%% supply the third, you must supply the second. Department defaults
%% to the department defined above and Institution defaults to Georgia
%% Institute of Technology.

\principaladvisor{Professor Henrik Christensen}
\committeechair{Professor Ignatius Arrogant}
\firstreader{Professor General Reference}[School of Mathematics]
\secondreader{Professor Ivory Insular}[Department of Computer Science and Operations Research][North Dakota State University]
\thirdreader{Professor Earl Grey}
\fourthreader{Professor John Smith}
\fifthreader{Professor Jane Doe}[Another Department With a Long Name][Another Institution]
%\setcounter{secnumdepth}{2}
\degree{Doctor of Philosophy}

%% Set \listmajortrue below, then uncomment and set this for
%% interdisciplinary PhD programs so that the title page says
%% ``[degree] in [major]'' and puts the department at the bottom of
%% the page, rather than saying ``[degree] in the [department]''

%% \major{Algorithms, Combinatorics, and Optimization} 

\copyrightyear{2010} 
\submitdate{August 2010} % Must be the month and year of graduation,
                         % not thesis approval! As of 2010, this means
                         % this text must be May, August, or December
                         % followed by the year.

%% The date the last committee member signs the thesis form. Printed
%% on the approval page.
\approveddate{1 July 2010}

\bibfiles{example-thesis}

%% The following are the defaults
%%    \titlepagetrue
%%    \signaturepagetrue
%%    \copyrightfalse
%%    \figurespagetrue
%%    \tablespagetrue
%%    \contentspagetrue
%%    \dedicationheadingfalse
%%    \bibpagetrue
%%    \thesisproposalfalse
%%    \strictmarginstrue
%%    \dissertationfalse
%%    \listmajorfalse
%%    \multivolumefalse

\begin{document}
\bibliographystyle{gatech-thesis}
%%
\begin{preliminary}
\begin{dedication}
\null\vfil
{\large
\begin{center}
To myself,\\\vspace{12pt}
Perry H. Disdainful,\\\vspace{12pt}
the only person worthy of my company.
\end{center}}
\vfil\null
\end{dedication}
\begin{preface}
Theses have elements.  Isn't that nice?
\end{preface}
\begin{acknowledgements}
I want to thank people
\end{acknowledgements}
% print table of contents, figures and tables here.
\contents
% if you need a "List of Symbols or Abbreviations" look into
% gatech-thesis-gloss.sty.
\begin{summary}
Why should I provide a summary?  Just read the thesis.
\end{summary}
\end{preliminary}
%%
\chapter{Introduction}

Introduction

\chapter{Map Annotation}

Map Annotation
  
\section{Related Work}

Related Work

\section{Semantic Maps}

Semantic Maps

\subsection{Waypoints}

\subsection{Planar Landmarks}

\subsection{Objects}


\section{User Interface}

User Interface


\section{Pointing Gestures for Human-Robot Interaction}

Pointing Gestures

\chapter{Navigation Among People}

Autonomous Robot Navigation

\section{Related Work}

Related Work

\section{State of Autonomous Robot Navigation}

State of Autonomous Robot Navigation

\section{Finding Goal Points for Navigation}

Finding Goal Points for Navigation

\section{People Aware Navigation}

People Aware Navigation

\section{Speed Maps for Safe Navigation}

Speed Maps for Safer Navigation

\chapter{Multimodal Person Tracking}

The ability to robustly track a person is an important prerequisite for human-robot interaction. To realize any task that involves humans, the challenge is the detection and tracking of humans in the vicinity of the robot considering the robot's movements, occlusions and robot's sensing capabilities. For mobile robot navigation, people of interest are either walking or standing, so we focus on detecting 

\section{Related Work}

Related Work

\section{Person Detection}

Person Detection

\subsection{Leg Detection}

Leg Detectionc  

\subsection{Torso Detection}

Leg Detection

\subsection{Lower Body Detection}

Lower Body Detection

\section{Person Tracking}

Multimodal Person Tracking

\section{Face Recognition}

Face Recognition

\chapter{Person Following}

Person Following

\section{Related Work}

Related Work

\section{Basic Person Following}

Basic Person Following

\section{Situation Aware Person Following}

Situation Aware Person Following

\subsection{Door Passing}

\subsection{User Activity Awareness}

\subsection{Corners}


\section{Application To Telepresence Robots}

Application To Telepresence Robots

\chapter{Person Guidance}

Person Guidance

\section{Related Work}

Related Work

\section{Guide Robot}

Guide Robot

\section{Application To Blind Users}

Application To Blind Users

\chapter{Conclusion}

Conclusion


\begin{table}
\caption{A table, centered.}
\begin{center}
\begin{tabular}{|l|r|}
  \hline 
Title & Author \\
\hline
War And Peace & Leo Tolstoy \\
The Great Gatsby & F. Scott Fitzgerald \\ \hline
\end{tabular}
\end{center}
\end{table}
%%


\nocite{*}
%% We need this since this file doesn't ACTUALLY \cite anything...
%%
\appendix
\chapter{QR Code Based Location Initialization}

QR Code Based Location Initialization

\chapter{Assisted Remote Control}

Assisted Remote Control

\chapter{Vibration Pattern Analysis for Haptic Belts}

Vibration Pattern Analysis for Haptic Belts


\begin{postliminary}
\references
\postfacesection{Index}{%
%%             ... generate an index here
%%         look into gatech-thesis-index.sty
}
\begin{vita}
Perry H. Disdainful was born in an insignificant town
whose only claim to fame is that it produced such a fine
specimen of a researcher.
\end{vita}
\end{postliminary}

\begin{abstract}
  This is the abstract that must be turned in as hard copy to the
  thesis office to meet the UMI requirements. It should \emph{not} be
  included when submitting your ETD. Comment out the abstract
  environment before submitting. It is recommended that you simply
  copy and paste the text you put in the summary environment into this
  environment. The title, your name, the page count, and your
  advisor's name will all be generated automatically.
\end{abstract}

\end{document}
