\chapter{Introduction}
\label{chapter:introduction}
Introduction

\section{Background}

\subsection{Social Spaces}

According to Lam~\cite{lam2011human}, mobile robots should obey certain rules while navigating in human environments. These rules include: not colliding anybody, not entering the personal space of a human unless the task is to approach the human and waiting if robot unwillingly enters the personal space of a human. Humans are already good at obeying such social conventions. Therefore most works on robot navigation in human environments is linked to human-human spatial interactions. One of the first studies in such interactions is conducted by Hall~\cite{hall1969hidden}. This study presents the proxemics theory, which categorizes the distance between people in four classes. These distances, named intimate, personal, social and public, provide spatial limits to different types of interactions. Kendon~\cite{kendon1990conducting}'s F-formation is based upon observations that people often group themselves in a spatial formation, e.g. in clusters, lines and circles. Some works adopted Hall distances and Kendon's formations for human-robot interaction. Huttenrauch~\cite{huttenrauch2006investigating} found that personal distance between a robot and a person varied in the range of 0.45 to 1.2 meters and but claimed that works of Hall and Kendon should be adapted to suit the dynamics of HRI. Avrunin~\cite{avrunin2013using} aims to learn acceptable distances from human-human experiments in an approaching scenario. 