\chapter{Introduction}
\label{chapter:introduction}
Introduction

\section{The Science of Personal Spaces}
\label{sec:personal_spaces}

Personal Spaces

\section{Home Tour Scenario}
\label{sec:home_tour_scenario}

Home Tour Scenario

\section{Background}

\subsection{Social Spaces}

According to Lam~\cite{lam2011human}, mobile robots should obey certain rules while navigating in human environments. These rules include: not colliding anybody, not entering the personal space of a human unless the task is to approach the human and waiting if robot unwillingly enters the personal space of a human. Humans are already good at obeying such social conventions. Therefore most works on robot navigation in human environments is linked to human-human spatial interactions. One of the first studies in such interactions is conducted by Hall~\cite{hall1969hidden}. This study presents the proxemics theory, which categorizes the distance between people in four classes. These distances, named intimate, personal, social and public, provide spatial limits to different types of interactions. Kendon~\cite{kendon1990conducting}'s F-formation is based upon observations that people often group themselves in a spatial formation, e.g. in clusters, lines and circles. Some works adopted Hall distances and Kendon's formations for human-robot interaction. Huttenrauch~\cite{huttenrauch2006investigating} found that personal distance between a robot and a person varied in the range of 0.45 to 1.2 meters and but claimed that works of Hall and Kendon should be adapted to suit the dynamics of HRI. Avrunin~\cite{avrunin2013using} aims to learn acceptable distances from human-human experiments in an approaching scenario. 


There are several areas of related research. The most
closely related approach to our own is that of Human
Augmented Mapping (HAM), introduced by Topp and Christensen
in [23] and [22]. The Human Augmented Mapping
approach is to have a human assist the robot in the mapping
process, and add semantic information to the map. The
proposed scenario is to have a human guide a robot on a tour
of an indoor environment, adding relevant labels to the map
throughout the tour. The HAM approach involves labeling
two types of entities: regions, and locations. Regions are
meant to represent areas such as rooms or hallways, and
serve as containers for locations. Locations are meant to
represent specific important places in the environment, such
as a position at which a robot should perform a task. This
approach was applied to the Cosy Explorer system, described
in [27], which includes a semantic mapping system that
multi-layered maps, including a metric feature based map, a
topological map, as well as detected objects. While the goal
of our approach is similar, we use a significantly different
map representation, method of labeling, and interaction.