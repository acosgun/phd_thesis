\begin{summary}

There are millions of robots in operation around the world today, and almost all of them operate on factory floors in isolation from people. However, it is now becoming clear that robots can provide much more value assisting people in daily tasks in human environments. Perhaps the most fundamental capability for a mobile robot is navigating from one location to another. Advances in mapping and motion planning research in the past decades made indoor navigation a commodity for mobile robots. Yet, questions remain on how the robots should move around humans. This thesis advocates the use of semantic maps and spatial rules of engagement to enable non-expert users to effortlessly interact with and control a mobile robot.

A core concept explored in this thesis is the Tour Scenario, where the task is to familiarize a mobile robot to a new environment after it is first shipped and unpacked in a home or office setting. During the tour, the robot follows the user and creates a semantic representation of the environment. The user labels objects, landmarks and locations by performing pointing gestures and using the robot's user interface. The spatial semantic information is meaningful to humans, as it allows providing commands to the robot such as ``bring me a cup from the kitchen table". While the robot is navigating towards the goal, it should not treat nearby humans as obstacles and should move in a socially acceptable manner.

Three main navigation behaviors are studied in this work. The first behavior is the point-to-point navigation. The navigation planner presented in this thesis borrows ideas from human-human spatial interactions, and takes into account personal spaces as well as reactions of people who are in close proximity to the trajectory of the robot. The second navigation behavior is person following. After the description of a basic following behavior, a user study on person following for telepresence robots is presented. Additionally, situation awareness for person following is demonstrated, where the robot facilitates tasks by predicting the intent of the user and utilizing the semantic map. The third behavior is person guidance. A tour-guide robot is presented with a particular application for visually impaired users.

\end{summary}