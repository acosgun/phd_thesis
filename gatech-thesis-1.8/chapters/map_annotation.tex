\chapter{Map Annotation}
\label{chapter:map_annotation}

Map Annotation


In mobile robotics, the standard practice for mapping and
localization is described as follows: When the robot is first
taken to a new environment, it has to map the environment.
There has been extensive research on Simultaneous Mapping
and Localization (SLAM) literature. The usual output is a
binary 2D grid map where 1’s represents an obstacle and
a 0’s represent free space. Once the map is created, the
robot can localize itself in the map while in operation.
Every time the robot is restarted, it has to start with an
initial estimation of its location. Although there are global
localization methods developed in the community, the usual
practice is that the robotics expert manually provides an
approximate initial location of the robot, then the localization
method corrects the localization estimation as the robot
moves in the environment.
  
\section{Related Work}


Related Work

\section{Semantic Maps}

Semantic Maps

\subsection{Waypoints}

\subsection{Planar Landmarks}

\subsection{Objects}


\section{User Interface}

User Interface


\section{Pointing Gestures for Human-Robot Interaction}

Pointing Gestures